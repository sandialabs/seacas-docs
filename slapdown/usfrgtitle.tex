\documentstyle[12pt]{tex$1520:xsand}
\makeatletter
\begin{document}
\begin{titlepage}   % Greg Sjaardema, 1521, 21 May 1987
 \let\footnotesize\normalsize  % Local definitions to make \thanks produce
 \let\@footnotenumber\fnsymbol % symbols for footnote numbers
 \thispagestyle{plain}
 \setcounter{page}{1}
\hbox{}\hfill{\parbox[t]{3.0in}{SAND87\up{--}2048C\\
\raggedright To be presented at the Joint FRG/USA Salt Near-Field Geochemistry
Workshop, Sept. 22--25, 1987, Albuquerque,~NM \par}}
\null
\vfil
\begin{center}
   {\Large\bf Numerical Modeling of Interactions Between Crushed and Intact
Salt\footnote{This work performed at Sandia National Laboratories
supported by the U.~S.~Dept. of Energy under contract number
DE-AC04-76DP00789.}\par}        
   \vskip .5truein                  % Vertical space after title.
   \large
   {G. D. Sjaardema and R. D. Krieg}\\      % Set author in \large size.
   {Applied Mechanics Division I}\\
   {Sandia National Laboratories}\\
   {Albuquerque, New Mexico\ \ 87185}
\end{center} \par
\vfil
\null
\large
\hbox to \hsize{\hfil Abstract\hfil}
\vskip 2ex
\normalsize
\noindent The results of several analyses studying the interaction
between intact salt and crushed salt in realistic field configurations,
such as backfilled shafts and drifts, will be presented.  The analyses
use a volumetric creep model for wet crushed salt and an
elastic-secondary creep model for the deviatoric behavior of intact
salt. The material parameters for the wet crushed salt model are
derived from hydrostatic consolidation tests performed on wet salt by
Holcomb.  The calculations show that the wet crushed salt does not
significantly retard the rate of closure of shafts and drifts until
the crushed salt is consolidated to approximately 95 percent of intact
salt density.  The importance of the crushed salt deviatoric behavior
will be discussed. 
\vfil
\null
\end{titlepage}
\end{document} 
